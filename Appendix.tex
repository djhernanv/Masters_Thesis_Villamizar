\thispagestyle{fancy}

\begin{appendices}

\chapter{Derivations}
\label{ax:derivations}

\section*{Investment Optimization in Tullock Contest}

A player $i$ wants to find the $I^*$ that maximizes his or her expected profit as expressed in the maximization problem:


\begin{equation}
    \underset{I_i}{\text{max}}\quad\mathbb{E}\Pi_i(I_i,I_j) = \frac{I_i}{I_i + \sum_{j\neq i}^n I_{j}}\mathbb{V} - I_i
\label{eq:exp_util_anex}
\end{equation}

Where $n$ is the  number of participants in the game.\\
Let us assume that the valuation of the game $\mathbb{V}$ is equal for all players and hence, that the optimal investment $I^{*}$ is equal for everyone. At the maximum of function \ref{eq:exp_util_anex} holds:
\begin{flalign*}
    \frac{d}{dI_i}(\frac{I_i\mathbb{V}}{I_i + \sum_{j\neq i}^n I_{j}} - I_i) = 0 &&
\end{flalign*}
which after deriving using the quotient rule results in:
\begin{flalign*}
    \frac{\mathbb{V}(I_i + \sum_{j\neq i}^n I_{j})-I_i\mathbb{V}}{(I_i + \sum_{j\neq i}^n I_{j})^2}-1&&
\end{flalign*}
Since we know that at the maximum $I_i^{*}=I_{-i}^{*}=I^{*}$ holds, we can simplify:
\begin{flalign*}
    \frac{n\mathbb{V}I^*-\mathbb{V}I^*}{(nI^*)^2}-1 = 0&&
\end{flalign*}
We further simplify to:
\begin{flalign*}
    \frac{\mathbb{V}\bcancel{I^*}(n-1)}{n^2I^{\bcancel{*2}}} = 1&&
\end{flalign*}
Which gives equation \ref{eq:opt_last}:
\begin{flalign*}
    I^{*} = \frac{n-1}{n^2}\mathbb{V}
\end{flalign*}

\chapter{Tables}

    \begin{figure}
        \centering
        \includegraphics[width=\textwidth]{graphs/Experimental_Design.pdf}
        \caption{Detailed structure of the experiment}
        \label{tab:exp_design}
    \end{figure}

\end{appendices}