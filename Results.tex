\chapter{Results}
\label{ch:results}
\thispagestyle{fancy}

All the data preparation, analysis and visualization was done using the statistical software \textit{R} \citep{rcoreteam2014}.\footnote{Packages used: \textit{tidyverse} \citep{wickham2017b}, \textit{data.table} \citep{dowle2018}, \textit{ineq} \citep{zeileis2014}, \textit{stargazer} \citep{hlavac2018}, \textit{ggplot2} \citep{wickham2016}, \textit{sjstats} \citep{ludecke2018}, \textit{texreg} \citep{leifeld2013}, \textit{ggthemes} \citep{arnold2018}, \textit{ggridges} \citep{wilke2018}, and others which are referenced in due course.} A total of 96 people (55 female and 41 male) with an average age of 25 years took part in a total of four sessions during the second half of 2018. Two of the sessions ran the control while the others two ran the treatment, such that exactly half the participants were in the treatment and half in the control.\\

At the end of the session, participants were paid anonymously, in private, and in cash, ranging from EUR 12.50 to 31.30 for an average payoff of EUR 22.90.\\  %\textcolor{red}{\textbf{To be discussed}: All data, scripts and results are available at XXX.XXX.XXX, even those that for space reasons were not included in the print version of this thesis.}

\section{Notes on the Methodology}

Although ANOVA models are commonly used in behavioral sciences when analyzing treatment differences in repeated measures, several authors have questioned their adequacy (\cite{camilli1987}, \cite{vasey1987}, \cite{jaeger2008}, \cite{locker2007}, \cite{krueger2004}). Especially critical in my design is the fact that participants are matched in groups which do not change during the experiment and may therefore develop their own dynamic. Imagine, for instance, a group in which the majority of participants are very risk friendly or have a high valuation. Investments might increase in total, even for those participants that are not as risk loving.\\ 
    
Instead, linear mixed models as used, for instance, by \cite{szaszi2018}, \cite{holsen2009} or \cite{yue2010}, are encouraged since they allow the study of the general effect of the variables of interest while allowing random intercepts for groups and participants and also whilst taking the auto-correlation of the repeated measure into account (\cite{galecki2013}, \cite{bolker2009}, \cite{mcculloch2015}, \cite{barr2013}, \cite{baayen2008}, \cite{fitzmaurice2015}). I therefore generally use a linear mixed model specification to be compared against a "fixed effects only" model in the results analysis below.\\ 
%Not included since there is no explicit model selection
%If not indicated otherwise, I chose the model by minimizing Akaike's Information Criterion which penalizes both a too simple and a too complex model.\\

\section{Preliminary Results}

Before diving into the hypotheses regarding investment and winning probabilities as postulated in section \ref{sec:hyp}, we want to get a sense of the general behavior patterns participants exhibited during the competition, particularly regarding the solution of the RET and how that behavior differed between treatments.

\subsection{Task Solving}
\label{sec:seq_prod}

During the benchmarking rounds, participants solved around 10 sequences and were paid one token per sequence. With the high piece rate of two tokens per sequence, participants solved an average of 15.3 sequences. The mean valuation, that is, the mean difference between the earnings (income from solving sequences plus income from the leisure mode) in the low and the high piece rate was of around 13 tokens. Table \ref{tab:avg_prod_bench} summarises these findings.

\begin{table}[!htbp] \centering 
  \caption{Mean Values of Production in Benchmark} 
  \label{tab:avg_prod_bench} 
\begin{tabular}{@{\extracolsep{5pt}} cccc} 
\\[-1.8ex]\hline 
\hline \\[-1.8ex] 
 & \makecell{avg. prod. \\ low wage (seqs)} & 
 \makecell{avg. prod. \\ high wage (seqs)} & mean valuation (tks)\\ 
\hline \\[-1.8ex] 
control & 9.77 & 15.29 & 12.9 \\ 
treatment & 10.25 & 15.29 & 13.09 \\ 
\hline \\[-1.8ex] 
\end{tabular} 
\end{table} 

During the competition, participants solved an average of 11.5 sequences and spent an average of 85.2 seconds in the ``switch'' mode. There was little difference in production between wages, i.e. between winners and losers, but some difference can be observed between treatments. Subjects in the treatment produced less on average and spent more time in the ``switch'' mode. Table \ref{tab:avg_prod} and Figure \ref{fig:production_boxplot} summarize these findings.\\

\begin{table}[!htbp] \centering
  \caption{Mean Values of Production\\
    \footnotesize{standard errors are reported in parentheses ()}} 
  \label{tab:avg_prod}
\begin{tabular}{@{\extracolsep{5pt}} cccc} 
\\[-1.8ex]\hline 
\hline \\[-1.8ex] 
 & wage & mean of solved sequences & mean of time spent in ``switch'' \\ 
\hline \\[-1.8ex] 
control & 1 & 12.02 & 78.66 \\ 
 &  & (0.23) & (3.13) \\ 
control & 2 & 11.91 & 79.09 \\
 &  & (0.3) & (4.22) \\ 
treatment & 1 & 11.1 & 92.28 \\
 &  & (0.21) & (2.86) \\ 
treatment & 2 & 11.12 & 90.23 \\
 &  & (0.33) & (4.42) \\ 
\hline \\[-1.8ex]
\end{tabular}
\end{table}  

\begin{figure}
    \centering
    \includegraphics[width=\textwidth]{graphs/production_boxplot.png}
    \caption{Number of Sequences Solved per Round and Treatment}
    \label{fig:production_boxplot}
\end{figure}
    
To see if the difference in production between the treatments is statistically significant, and to see how other factors play a role, I used a linear mixed model (LMM) with random intercepts evaluated using the \textit{R} package \textit{lme4} \citep{bates2015}. Since the design only allows for between-group comparisons in the treatment, I do not include a random slope per participant, assuming thus that the effect of time will be the same for all subjects.\\

In this model, the fixed effects are: the \textit{treatment}, \textit{was winner} (if a participant won the previous round and is playing with the high wage), and the \textit{round} number. The random effect is the \textit{subjects id} nested within the \textit{group}. Results are largely reported following \cite{barr2013} and are summarized in table \ref{table:lmer_prod} in the \textit{sequence production} column. P-values were calculated using the likelihood ratio test of the \textit{R} package \textit{lmerTest} \citep{kuznetsova2017}.\\

\textit{Treatment} has a small, though significant, effect at the 0.1 level with subjects in the control group solving just shy of 1 sequence more per round (+0.872). All the other fixed effects have no significant or large effect with the model in total explaining 42\% of the observed variance in which the fixed effects explain 1.52\% (conditional R-squared and marginal R-squared respectively, calculated using the \textit{psycho} package \citep{makowski2018}). The variables with no effect include, interestingly, the wage at which participants played.\\

%Not included since Loy argues not necessary
%Note that although the residual plot and QQ-plot do not make apparent any deviation from a normal distribution nor any systematic increase or decrease in variance, there is much debate as to how to appropriately check for the validity of assumptions in LMMs  \citep{loy2017}, specially when dealing with distributions other than normal, and will therefore not be addressed here in this thesis.\\

It is important to remember that in this experimental design, payoffs are not exclusively determined by the amount of sequences solved. They depend also on: the time spent in the ``switch'' mode and therefore on the optimal change to it, the invested amount in human capital, and, in the treatment, how much the other players worked. The boxplot in Figure \ref{fig:earnings_boxplot} seems to suggest that the net income is higher in the treatment groups and that, apart from the transition from round 1 to round 2, it does not change over time. Note that, here, net income describes income after taxation \textit{and} redistribution, but before investment. The variance of the treatment is smaller since, in that group, by definition, net incomes are closer to each other.\\

In order to see if there were significant differences in net income between treatments, I use a model with fixed effects (\textit{treatment, was winner} and \textit{round}) and summarise the findings in the \textit{net income} column of table \ref{table:lmer_prod}.\\

\begin{table}[!htbp] \centering 
  \caption{Linear Mixed Model - Sequence Production and Net Income} 
  \label{table:lmer_prod} 
  \scalebox{0.8}{%
\begin{tabular}{@{\extracolsep{5pt}}lcc} 
\\[-1.8ex]\hline 
\hline \\[-1.8ex] 
 & \multicolumn{2}{c}{\textit{Dependent Variable:}}\\ 
\cline{2-3} 
\\[-1.8ex] & Sequence Production & Net Income \\ 
\hline \\[-1.8ex] 
 treatment & $-$0.872$^{*}$ & 0.598 \\ 
  & (0.499) & (0.471)\\ 
  & \\ 
 was winner & 0.028 & 5.977$^{***}$\\ 
  & (0.231) & (0.289)\\ 
  & \\ 
 round & $-$0.003 & 0.090$^{*}$\\ 
  & (0.043) & (0.054) \\ 
  & \\ 
 Constant & 11.985$^{***}$ & 21.434$^{***}$\\ 
  & (0.408) & (0.422)\\ 
  & &\\ 
\hline \\[-1.8ex] 
Observations & 768 & 768 \\ 
Log Likelihood & $-$1,943.528 & $-2,098.867$\\ 
Akaike Inf. Crit. & 3,901.055 & $4,211.735$\\ 
Bayesian Inf. Crit. & 3,933.525 & $4,244.241$\\
\hline \\ [-1.8ex] 
Var: participant|group (Intercept) & 5.08 & 3.86\\
Var: group (Intercept) & 0.00 & 0.00\\
Var: Residual & 7.28 & 11.77\\
\hline \\[-1.8ex] 
\hline 
\textit{Note:}  & &\multicolumn{1}{r}{$^{*}$p$<$0.1; $^{**}$p$<$0.05; $^{***}$p$<$0.01} \\ 
\end{tabular}
}
\end{table} 

Winning the previous round, i.e. playing with the high wage has a strong and significant effect on net income while time, although marginally significant, has only a small effect. The introduction of a tax and redistribution scheme, on the other hand, does not have a significant effect. The model helps explaining 50.11\% (restricted R-squared) of the variance with the fixed effects accounting for 33.73\% (marginal R-squared) of the variance.\\

Notice that while the taxation treatment has a negative impact on the number of sequences solved, it has no effect on net income. This is an indication that participants were over-performing in the control and, when confronted with a redistribution scheme, diminished their production such that the moment of switching to the leisure mode was closer to the optimum. Thus, they earned relatively equal, despite the effective tax rate of 40\%.\\

\begin{figure}
    \centering
    \includegraphics[width=\textwidth]{graphs/earnings_boxplot.png}
    \caption{Income after Taxation and Redistribution, and Before Investment}
    \label{fig:earnings_boxplot}
\end{figure}


%\begin{table}[!htbp] \centering 
%  \caption{Linear Mixed Model - Net Income} 
%  \label{table:earnings_lmer}
%  \scalebox{0.8}{%
%\begin{tabular}{@{\extracolsep{5pt}}lc} 
%\\[-1.8ex]\hline 
%\hline \\[-1.8ex] 
% & \multicolumn{1}{c}{\textit{Dependent variable:}} \\ 
%\cline{2-2} 
%\\[-1.8ex] & net income \\ 
%\hline \\[-1.8ex] 
% treatment & 0.598 \\ 
%  & (0.471) \\ 
%  & \\ 
% was winner & 5.977$^{***}$ \\ 
%  & (0.289) \\ 
%  & \\ 
% round & 0.090$^{*}$ \\ 
%  & (0.054) \\ 
%  & \\ 
% Constant & 21.434$^{***}$ \\ 
%  & (0.422) \\ 
%  & \\ 
%\hline \\[-1.8ex] 
%Observations & 768 \\ 
%Log Likelihood & $-$2,098.867 \\ 
%Akaike Inf. Crit. & 4,211.735 \\ 
%Bayesian Inf. Crit. & 4,244.241 \\ 
%\hline 
%Var: participant|group (Intercept) & 3.86          \\
%Var: group (Intercept)           & 0.00          \\
%Var: Residual                              & 11.77         \\
%\hline
%\hline \\[-1.8ex] 
%\textit{Note:}  & \multicolumn{1}{r}{$^{*}$p$<$0.1; $^{**}$p$<$0.05; $^{***}$p$<$0.01} \\ 
%\end{tabular}
%}
%\end{table} 


\subsection{Switching Time}
During the competition, participants appear willing to work more than their optimum in order to achieve a higher available income. Remember, participants should switch as soon as the time they need to solve a given task surpasses the earnings in the ``switch'' mode for the corresponding time. In the control group, players with the low wage should change as soon as they take longer than 10 seconds to solve the task and, in the treatment, as soon as they need more than six seconds. In each treatment, players with the high wage should switch at twice the respective time. Figure \ref{fig:time_per_task} shows the time needed for each task per round and marks the 10 and 20 seconds cut-offs.\\

It seems as if, in fact, the time needed to solve each sequence is more compressed to the middle in the control group. That is, fewer people are taking time above the cut-offs to solve sequences before switching. Table \ref{table:lmer_prod} offers a strong indication in this direction, as explained at the end of section \ref{sec:seq_prod}. However, a Wilcoxon test\footnote{This is conducted using the \textit{stats} package \cite{rcoreteam2014}.} cannot reject the null hypothesis (p = 0 .283) of no shift with regard to the overall seconds over the 10 or 20 cut-offs. In chapter \ref{ch:discussion}, I address a design change that could make it easier to identify if there are general changes to the switching point when taxation is introduced.\\

\begin{figure}
    \centering
    \includegraphics[width=\textwidth]{graphs/time_task_grid.png}
    \caption{Seconds Needed to Solve Sequences}
    \label{fig:time_per_task}
\end{figure}

\subsection{Controls}

In the \textit{Cognitive Reflection Test}, participants answered, out of a maximum of four, on average 2.22 (s.e. 0.1) questions correctly. Men answered, on average, 0.3 (s.e. 0.15) more questions correctly, with women giving 2.09 (s.e. 0.13) correct answers on average.\\

In the \textit{Risk Elicitation} task, participants, on average, decided to switch on row 6.6, which according to table \ref{table:HL} indicates a risk averse to very risk averse behaviour. Men generally showed a more risk averse behaviour, switching on average at row 6.93 (s.e. 0.32), while women switched on average already at row 6.36 (s.e. 0.33). Figure \ref{fig:hist_mpl} shows a histogram of the row at which participants switched.\\

\begin{figure}
    \centering
    \includegraphics[scale=0.3]{graphs/hist_mpl.png}
    \caption{Histogram of Choice of Switching Row}
    \label{fig:hist_mpl}
\end{figure}


Participants did not seem to change their views on the fairness of the game. Figure \ref{fig:fairness_boxplot} shows a boxplot of their assessment at the beginning and at the end of the competition rounds with 0 being only luck and 7 being only skill that determined the outcome of the game. Table \ref{tab:fair_ols} lists the results of a linear regression. However, no characteristic shows any significant effect on the change in fairness sentiment.\\

\begin{figure}
    \centering
    \includegraphics[width=\textwidth]{graphs/fairness_sentiment_boxplot.png}
    \caption{Fairness Sentiment Voting}
    \label{fig:fairness_boxplot}
\end{figure}



\begin{table}[!htbp] \centering 
  \caption{OLS Change in Fairness Sentiment} 
  \label{tab:fair_ols} 
   \scalebox{0.6}{%
\begin{tabular}{@{\extracolsep{5pt}}lc} 
\\[-1.8ex]\hline 
\hline \\[-1.8ex] 
\\[-1.8ex] & Change in Fairness Sentiment \\ 
\hline \\[-1.8ex] 
 treatment & $-$0.183 \\ 
  & (0.391) \\ 
  & \\ 
 payoff & 0.015 \\ 
  & (0.052) \\ 
  & \\ 
 CRT score & 0.063 \\ 
  & (0.188) \\ 
  & \\ 
 risk aversion & 0.013 \\ 
  & (0.084) \\ 
  & \\ 
 gender (male) & $-$0.181 \\ 
  & (0.366) \\ 
  & \\ 
 valuation & 0.048 \\ 
  & (0.063) \\ 
  & \\ 
 Constant & $-$0.971 \\ 
  & (1.298) \\ 
  & \\ 
  \hline
Observations & 96 \\ 
R$^{2}$ & 0.016 \\ 
Adjusted R$^{2}$ & $-$0.051 \\ 
Residual Std. Error & 1.733 (df = 89) \\ 
F Statistic & 0.235 (df = 6; 89) \\ 
\hline \\[-1.8ex] 
\hline 
\textit{Notes:} & \multicolumn{1}{l}{$^{*}$p$<$0.1; $^{**}$p$<$0.05; $^{***}$p$<$0.01} \\ 
% & \multicolumn{1}{l}{$^{**}$Significant at the 5 percent level.} \\ 
% & \multicolumn{1}{l}{$^{*}$Significant at the 10 percent level.} \\ 
\end{tabular} 
}
\end{table} 

\section{Hypotheses Testing}

In this section, we will look at the results regarding, first, the distribution of probabilities and its change across rounds, and, second, the factors affecting the probability of winning for a given individual according to the hypotheses postulated in section \ref{sec:hyp}.


\subsection{Hypotheses 1 and 2}

\textbf{Hypothesis \ref{hyp:treat}:} \textit{Mean investments will be higher than the expected profit maximizing value for all treatments, but will decrease over time.}\\ 
\textbf{Hypothesis \ref{hyp:treat-overinvest}:} \textit{Participants in the treatment group will invest a smaller multiple of their optimal investment.}\\

The first step taken to test these hypotheses is to determine the optimal investment value for each participant. Running a t-test on the hypothesis that the distance to the mean is 0, in other words, that there is no difference in valuation between participants, supports this assumption (p = 1 for control and p = 0.85 for the treatment). This result greatly simplifies the process by allowing the use of formula \ref{eq:opt_last}.\\

Furthermore, since the treatment and control groups have different valuations, to more easily compare the results I analyze the mean difference over optimal investment for both groups instead of the nominal amount of over-investment. Note that in comparison to table \ref{tab:avg_prod_bench}, the valuation for the treated groups is calculated from the difference in the taxed benchmark rounds as explained in section \ref{ss:tax_bench}.\\

Consistent with expectations, participants bid on average significantly more than their optimum but decreasingly so over time, as Figure \ref{fig:over_invest} shows. It plots each round to the mean per treatment of the over investment ratio, i.e., how many times more than the optimum was invested on average.\\

Interestingly, participants in the taxation treatment invest both nominally and relatively more than those in the control, with treatment participants bidding on average 5 and a half times their optimal investment in the first round and 2.3 times their optimum in the last round. Participants in the control, on the other hand, invest from 1.93 times their optimum in the first round, to 1.36 times in the last round.\\


\begin{figure}
    \centering
    \includegraphics[width=\textwidth]{graphs/over_invest.png}
    \caption{Ratio of Over Investment per Round}
    \label{fig:over_invest}
\end{figure}


A possibility for why participants in the treatment invest relatively more is that subjects in both groups apply a simple heuristic of a given proportion of available income to invest. Given the fact that taxed players have a lower valuation but a higher disposable income, their bids are a higher proportion of the optimum. Running a regression on the investment ratio of available income in fact deletes the effect of all fixed effects but time, suggesting that participants tend to invest the same amount of their available income, regardless of their earnings or idiosyncratic features. Figure \ref{fig:invest_share} shows a plot of the mean shares of the available income invested. The shares decrease with time in similar fashion for both treatments and, apart from the final investing round, within a standard error from each other.\\

\begin{figure}[h!]
    \centering
    \includegraphics[width=\textwidth]{graphs/investment_share_geom_line.png}
    \caption{Investment Share of Available Income per Round}
    \label{fig:invest_share}
\end{figure}

Table \ref{tab:over_invest} presents a linear mixed model dissecting the most important causes over-investing. The random effects are \textit{participants} nested in \textit{group}s.\\

\begin{table} \centering 
  \caption{Linear Mixed Model: Over Investment Ratio} 
  \label{tab:over_invest}
  \scalebox{0.6}{%
\begin{tabular}{@{\extracolsep{5pt}}lcc} 
\\[-1.8ex]\hline 
\hline \\[-1.8ex] 
\\[-1.8ex] & \multicolumn{2}{c}{Ratio of Investment Over Optimum} \\ 
\\[-1.8ex] & (1) & (2)\\ 
\hline \\[-1.8ex] 
 treatment & 2.314$^{***}$ & 2.075$^{***}$ \\ 
  & (0.698) & (0.678) \\ 
  & & \\ 
 round & $-$0.307$^{***}$ & $-$0.307$^{***}$ \\ 
  & (0.055) & (0.055) \\ 
  & & \\ 
 was winner & 0.256 & 0.281 \\ 
  & (0.271) & (0.271) \\ 
  & & \\ 
 CRT score &  & $-$0.183 \\ 
  &  & (0.289) \\ 
  & & \\ 
 risk aversion &  & $-$0.375$^{***}$ \\ 
  &  & (0.126) \\ 
  & & \\ 
 gender (Male) &  & 0.284 \\ 
  &  & (0.583) \\ 
  & & \\ 
 Constant & 2.813$^{***}$ & 5.689$^{***}$ \\ 
  & (0.546) & (1.097) \\ 
  & & \\ 
  \hline
  \hline
Observations & 672 & 672 \\ 
Log Likelihood & $-$1,751.726 & $-$1,747.753 \\ 
Akaike Inf. Crit. & 3,517.453 & 3,515.506 \\ 
Bayesian Inf. Crit. & 3,549.024 & 3,560.608 \\
\hline
Var: participant|group (Intercept) & 5.66 &     5.17     \\
Var: group (Intercept)           & 1.62  &    1.48    \\
Var: Residual                              & 8.14   &   8.14    \\
\hline \\[-1.8ex] 
\textit{Notes:} & \multicolumn{2}{r}{$^{*}$p$<$0.1; $^{**}$p$<$0.05; $^{***}$p$<$0.01} \\ 
% & \multicolumn{2}{l}{$^{**}$Significant at the %5 percent level.} \\ 
% & \multicolumn{2}{l}{$^{*}$Significant at the %10 percent level.} \\ 
\end{tabular}
}
\end{table}


The first and second models have a total explanatory power of around 52\% (conditional R-squared) in which the fixed effects of the first explain 10.03\% and the second 14.5\% of the variance (marginal R-squared). The effect of the treatment is significant and large for both models with participants investing, contrary to hypothesis \ref{hyp:treat-overinvest}, almost double as much over the optimum as subjects in the control group. Round repetition and risk averse behavior have both a significant, although relatively small, effect in the expected negative direction. The CRT score and whether a participant was playing with a high wage did not have a significant effect on the over-investment ratio.\\

Figure \ref{fig:beliefs_smooth} plots the differences between the estimation made by participants before the investment round and the actual investments  made. In the plot, a reduction in the mean over-estimation can be seen. It indicates better knowledge of other's behaviour with progressing rounds. It is nevertheless surprising how little accuracy participants appear to gain even after playing seven rounds and having been shown all relevant information.\\

\begin{figure}
    \centering
    \includegraphics[width=\textwidth]{graphs/beliefs_smooth_lm.png}
    \caption{Mean Absolute Over or Under-estimation of Other's Investments by Group and Round}
    \label{fig:beliefs_smooth}
\end{figure}

A Wilcoxon rank test (p = 0.01) suggests, furthermore, that total welfare (the total amount of tokens paid out during competition) is larger in the treatment than in the control. Higher welfare in the treatment seems to be driven by working closer to the optimum.\\ 
%Adding the Gini coefficient of the available income to the model in table \ref{tab:over_invest} does not show any significance of inequality nor does it change the remaining coefficients which strengthens the idea that the higher investment in the treatment is due to an available income share heuristic.\\

\subsection{Hypothesis 3}

\textbf{Hypothesis \ref{hyp:wins}:} \textit{The coefficient of variation of winning probabilities will be increasing across rounds while being smaller in the treatment group.}\\

Participants appear to invest less with each round as the histogram in Figure \ref{fig:invest_hist} shows. Indeed, several participants do not invest anything at all and their number increases with each repetition which leads to the pattern shown in Figure \ref{fig:invest_prob_point}: With increasing rounds, the relationship between the invested amounts and the probability obtained grow apart. This means, in particular, that with each passing round, more lower investments had a higher probability of winning.\\

\begin{figure}
    \centering
    \includegraphics[width=\textwidth]{graphs/investment_amount_hist.png}
    \caption{Histogram of Investment Amounts by Round. The Red Line Demarks the Median of the Distribution.}
    \label{fig:invest_hist}
\end{figure}

\begin{figure}
    \centering
    \includegraphics[width=\textwidth]{graphs/invest_prob_point.png}
    \caption{Plot of the Probability Achieved by a Given Investment per Round}
    \label{fig:invest_prob_point}
\end{figure}

In terms of probabilities, I am interested in how their distribution change. Naturally, the mean probability of earning the high wage is always equal to $1/n$, but its distribution can vary from exactly $1/n$ for each participant, to only one person in the group having a probability of 1 and all the rest a probability of 0. Figure \ref{fig:dens_prob} shows the density plots for each round. A completely equal distribution would have its maximum at 0.33 with no values around it, while a completely unequal distribution would see two peaks at each extreme with the one at 0 being twice the size of the the one at 1.\\

\begin{figure}[H]
    \centering
    \includegraphics[width = \textwidth]{graphs/density_ridge_prob.png}
    \caption{Density Plots of Probability across Rounds}
    \label{fig:dens_prob}
\end{figure}

Figure \ref{fig:dens_prob} seems, in fact, to show an increase of density at the ends of the distribution with round 1 having a rather flat shape and round 7 stronger peaks at both 0 and 1 probability. Figure \ref{fig:var_coeff_boxplot} shows a box plot of the coefficient of variations across rounds for both treatments. In fact, it seems to increase across rounds but not significantly between treatments.\\

\begin{figure}[H]
    \centering
    \includegraphics[width = 0.8\textwidth]{graphs/var_coeff_prob_boxplot.png}
    \caption{Boxplot Coefficient of Variation}
    \label{fig:var_coeff_boxplot}
\end{figure}

Table \ref{tab:var_coeff_ols} shows the results for an OLS regression of two models explaining the difference across groups in the coefficient of variation for the probabilities of winning. A linear mixed model is left out here since each group is part of only one treatment and the variables are all aggregated at the group level. Only round repetition appears to have a significant, although small effect (in our setting, the coefficient of variance can fluctuate between 0 and 1.73) on the coefficient of variation. Idiosyncratic group values like the mean value of valuation, CRT score or risk aversion do not play a significant role in the variation of probabilities. Taxation and redistribution also do not play a role. In other words, introducing a redistribution scheme does not avoid the fact that, with time, the distribution of probabilities will go to extremes with some participants having most of the chance to win and the rest having almost none.\\

\begin{table}[!htbp] \centering 
  \caption{OLS Coefficient of Variation} 
  \label{tab:var_coeff_ols} 
\begin{tabular}{@{\extracolsep{5pt}}lcc} 
\\[-1.8ex]\hline 
\hline \\[-1.8ex] 
\\[-1.8ex] & \multicolumn{2}{c}{Coefficient of Variation} \\ 
\\[-1.8ex] & (1) & (2)\\ 
\hline \\[-1.8ex] 
 treatment & $-$0.025 & $-$0.031 \\ 
  & (0.063) & (0.065) \\ 
  & & \\ 
 round & 0.043$^{***}$ & 0.043$^{***}$ \\ 
  & (0.016) & (0.016) \\ 
  & & \\ 
 mean CRT score &  & 0.092 \\ 
  &  & (0.059) \\ 
  & & \\ 
 mean risk aversion &  & $-$0.020 \\ 
  &  & (0.025) \\ 
  & & \\ 
 mean valuation &  & $-$0.033 \\ 
  &  & (0.023) \\ 
  & & \\ 
 Constant & 0.856$^{***}$ & 1.218$^{***}$ \\ 
  & (0.077) & (0.338) \\ 
  & & \\
  \hline
Observations & 224 & 224 \\ 
R$^{2}$ & 0.034 & 0.053 \\ 
Adjusted R$^{2}$ & 0.025 & 0.031 \\ 
Residual Std. Error & 0.468 (df = 221) & 0.467 (df = 218) \\ 
F Statistic & 3.857$^{**}$ (df = 2; 221) & 2.431$^{**}$ (df = 5; 218) \\ 
\hline
\hline \\[-1.8ex] 
\textit{Notes:} & \multicolumn{2}{r}{$^{*}$p$<$0.1; $^{**}$p$<$0.05; $^{***}$p$<$0.01} \\ 
% & \multicolumn{2}{l}{$^{**}$Significant at %the 5 percent level.} \\ 
% & \multicolumn{2}{l}{$^{*}$Significant at the %10 percent level.} \\ 
\end{tabular} 
\end{table}


Of course, this does not tell us much about what determines the probability of a given participant to win a contest. To analyze it, I suggest again a linear mixed model, although this time a generalized mixed model from a beta distribution as the results represent a distribution of probabilities with values restricted from 0 to 1\citep[Chapter~17]{krishnamoorthy2016}.\footnote{Since several participants did not invest at all and thus had a probability of winning of 0, this led to some other participants having a probability of winning of 1. Those values were transformed to 0.0000000000001 and 0.0000000000009 respectively to allow them to fit a beta distribution. The analysis was conducted with the \textit{glmmTMB} package \citep{brooks2017}.} Table \ref{tab:glm_prob} shows the results.\\

The only variable that shows a significant, although small, effect on the probability of winning is the size of the available income. Note, in particular, that having won alone has no significant effect on the probability of winning. This is in part because participants who are working with the lower rate can still achieve a large available income if they switch above their optimal point and thus forfeit time in the leisure mode. An alternative model including cumulative wins also showed no influence on the probability of winning.\footnote{The regression table is in the Appendix \ref{ax:glm_prob}.}\\

\begin{table}[!htbp] \centering 
  \caption{GLMM Probability of Winning} 
  \label{tab:glm_prob} 
  \scalebox{0.6}{%
\begin{tabular}{@{\extracolsep{5pt}}lcc} 
\\[-1.8ex]\hline 
\hline \\[-1.8ex] 
\\[-1.8ex] & \multicolumn{2}{c}{Probability of Winning} \\ 
\\[-1.8ex] & (1) & (2)\\ 
\hline \\[-1.8ex] 
 treatment & $-$0.139 &  $-$0.087241 \\ 
  & (0.22) & (0.218) \\ 
  & & \\ 
 available income & 0.045$^{***}$ & 0.045$^{***}$ \\ 
  & (0.008) & (0.008) \\ 
  & & \\ 
 was winner & $-$0.084 &  $-$0.083 \\ 
  & (0.119) & (0.119) \\ 
  & & \\ 
 gender (male) &  & 0.356 \\ 
  &  & (0.223) \\ 
  & & \\ 
 CRT score &  & $-$0.051 \\ 
  &  & (0.113) \\ 
  & & \\ 
 risk aversion & & $-$0.034 \\ 
  &  & (0.049) \\ 
  & & \\ 
 valuation &  & 0.056 \\ 
  &  & (0.038) \\ 
  & & \\ 
 mean investment belief &  & 0.003 \\ 
  &  & (0.008) \\ 
  & & \\ 
 Constant & $-$1.413$^{***}$ & $-$2.454$^{***}$ \\ 
  & (0.209) & (0.647) \\ 
  & & \\ 
  \hline
Observations & 672 & 672 \\ 
Log Likelihood & 3798.4 & 3801.0 \\ 
Akaike Inf. Crit. & $-$7580.9 & $-$7575.9 \\ 
Bayesian Inf. Crit. & $-$7544.8 & $-$7517.3 \\
\hline
Num. groups: participant:group     & 96     &   96   \\
Num. groups: group               & 32      &    32  \\
Num. groups: round                         & 7   & 7        \\
\hline
Var: participant:group (Intercept) & 0.93   & 0.02     \\
Var: group (Intercept)           & 0.00   & 0.00     \\
Var: round (Intercept)           & 0.00    & 0.00    \\
Var: Residual                    & 0.00     &  0.02  \\
\hline
\hline \\[-1.8ex] 
\textit{Notes:} & \multicolumn{2}{l}{$^{*}$p$<$0.1; $^{**}$p$<$0.05; $^{***}$p$<$0.01} \\ 
% & \multicolumn{2}{l}{$^{**}$Significant at %the 5 percent level.} \\ 
% & \multicolumn{2}{l}{$^{*}$Significant at the %10 percent level.} \\ 
\end{tabular} 
}
\end{table} 