\chapter{Results}
\label{ch:results}
\thispagestyle{fancy}

All the data preparation, analysis and visualization was done using the statistical software \textit{R} \citep{rcoreteam2014}\footnote{and the packages \textit{tidyverse} \citep{wickham2017b}, \textit{data.table} \citep{dowle2018}, \textit{ineq} \citep{zeileis2014}, \textit{stringer} \citep{wickham2018}, \textit{sjstats} \citep{ludecke2018}, and others which are referenced in due course.}. 
All data and scripts are available at XXX.XXX.XXX

\section{Descriptive Statistics and General Results}

Participants solved an average of 11.5 sequences in the competition rounds and spent an average of 85.2 seconds in the \textit{switch} mode. There was little difference in production between wages, i.e. between winners and losers, but some difference can be observed between treatments. Subjects in the treatment produced less in average and spent more time in the \textit{switch} mode. Table \ref{tab:avg_prod} and graph \ref{fig:production_boxplot} summarize these findings.\\

\begin{table}[!htbp] \centering
  \caption{Mean Values of Production\\
    \footnotesize{standard errors are reported in parentheses ()}} 
  \label{tab:avg_prod}
\begin{tabular}{@{\extracolsep{5pt}} cccc} 
\\[-1.8ex]\hline 
\hline \\[-1.8ex] 
treatment & wage & mean solved sequences & mean of time spent in switch \\ 
\hline \\[-1.8ex] 
0 & 1 & 12.02 & 78.66 \\ 
 &  & (0.23) & (3.13) \\ 
0 & 2 & 11.91 & 79.09 \\
 &  & (0.3) & (4.22) \\ 
1 & 1 & 11.1 & 92.28 \\
 &  & (0.21) & (2.86) \\ 
1 & 2 & 11.12 & 90.23 \\
 &  & (0.33) & (4.42) \\ 
\hline \\[-1.8ex]
\end{tabular}
\end{table}  

\begin{figure}
    \centering
    \includegraphics[width=\textwidth]{graphs/production_boxplot.png}
    \caption{Number of Sequences Solved per Round and Treatment}
    \label{fig:production_boxplot}
\end{figure}

\subsection{Sequence Production \& Income}
To see if the difference in production between the treatments is statistically significant, and to see how other factors play a role, I used a linear mixed model with random intercepts evaluated using the \textit{R} package \textit{lmer} \citep{bates2015}. I chose this model since it offers more flexibility than a traditional ANOVA analysis. Specially important for my design is the fact that it allows to have a random intercept for each subject and nesting in the group, since we might expect that certain participants tend to work and invest more than others and that that affects the overall performance of the group. Since the design allows only for between-groups comparisons regarding the treatment, I exclude a random slope per participant, assuming thus that the effect of time will be equivalent for all subjects.\\

In this model, the fixed effects are the \textit{treatment}, \textit{was winner} (if a participant won the previous round and is playing with the high wage), and the \textit{round} number. The random effect is the \textit{subjects id} nested within the \textit{group}. Results are reported following largely \cite{barr2013} and are summarized in table \ref{table:lmer_prod}. P-values were calculated using the likelihood ratio test.\\

The residual plot and qq-plot do not make apparent any deviation from a normal distribution nor any systematic increase or decrease in variance. \textit{Treatment} has a small although significant effect at the 0.1 level with subjects in the control group solving just shy of 1 sequence more. All the other fixed effects have no significant or large effect with the model in total explaining 42\% of the observed variance (conditional R-squared) in which the fixed effects explain 1.52\% (marginal r-squared). This includes, interestingly, the wage at which participants played.\\

\begin{table}[!htbp] \centering 
  \caption{Linear Mixed Model - Sequence Production} 
  \label{table:lmer_prod} 
  \scalebox{0.8}{%
\begin{tabular}{@{\extracolsep{5pt}}lc} 
\\[-1.8ex]\hline 
\hline \\[-1.8ex] 
 & \multicolumn{1}{c}{\textit{Dependent variable:}} \\ 
\cline{2-2} 
\\[-1.8ex] & Sequence Production \\ 
\hline \\[-1.8ex] 
 treatment & $-$0.872$^{*}$ \\ 
  & (0.499) \\ 
  & \\ 
 was winner & 0.028 \\ 
  & (0.231) \\ 
  & \\ 
 round & $-$0.003 \\ 
  & (0.043) \\ 
  & \\ 
 Constant & 11.985$^{***}$ \\ 
  & (0.408) \\ 
  & \\ 
\hline \\[-1.8ex] 
Observations & 768 \\ 
Log Likelihood & $-$1,943.528 \\ 
Akaike Inf. Crit. & 3,901.055 \\ 
Bayesian Inf. Crit. & 3,933.525 \\
\hline \\ [-1.8ex] 
Var: participant|group (Intercept) & 5.08\\
Var: group (Intercept) & 0.00 \\
Var: Residual & 7.28 \\
\hline \\[-1.8ex] 
\hline 
\textit{Note:}  & \multicolumn{1}{r}{$^{*}$p$<$0.1; $^{**}$p$<$0.05; $^{***}$p$<$0.01} \\ 
\end{tabular}
}
\end{table} 

In this experimental design, earnings are not exclusively determined by the amount of sequences solved. It depends also on time spent in the \textit{switch} mode and therefore on the optimal change to it; the invested amount and, in the treatment, how much the other players worked. The boxplot in figure \ref{fig:earnings_boxplot} seems to suggest that net income is higher in the treatment groups and that, apart from round 1, they do not change over time. Note that net income describes here income after taxation AND redistribution but before investment. To see if there were differences are significant, I use a model similar to the one explained above. With fixed effects \textit{treatment, was winner} and \textit{round}. Table \ref{table:earnings_lmer} summarizes the findings.\\

Winning the previous round, i.e. playing with the high wage has a strong and significant effect on net income while time, although slightly significant, has only a small effect. The introduction of a tax and redistribution, on the other hand, does not have a significant effect. The model helps explaining 50.11\% (restricted r-squared) of the variance with the fixed effects accounting for 33.73\% (marginal r-squared) of the variance. The residual plot and qq-plot do not make apparent any deviation from a normal distribution nor any systematic increase or decrease in variance.\\

A discussion about why the treatment fails to show an impact on net income is well beyond the scope of this thesis but it is addressed in the \textit{further research} section in chapter \ref{ch:conclusion}.


\begin{figure}
    \centering
    \includegraphics[width=\textwidth]{graphs/earnings_boxplot.png}
    \caption{Income after Taxation and Redistribution, and before investment}
    \label{fig:earnings_boxplot}
\end{figure}


\begin{table}[!htbp] \centering 
  \caption{Linear Mixed Model - Net Income} 
  \label{table:earnings_lmer}
  \scalebox{0.8}{%
\begin{tabular}{@{\extracolsep{5pt}}lc} 
\\[-1.8ex]\hline 
\hline \\[-1.8ex] 
 & \multicolumn{1}{c}{\textit{Dependent variable:}} \\ 
\cline{2-2} 
\\[-1.8ex] & net income \\ 
\hline \\[-1.8ex] 
 treatment & 0.598 \\ 
  & (0.471) \\ 
  & \\ 
 was winner & 5.977$^{***}$ \\ 
  & (0.289) \\ 
  & \\ 
 round & 0.090$^{*}$ \\ 
  & (0.054) \\ 
  & \\ 
 Constant & 21.434$^{***}$ \\ 
  & (0.422) \\ 
  & \\ 
\hline \\[-1.8ex] 
Observations & 768 \\ 
Log Likelihood & $-$2,098.867 \\ 
Akaike Inf. Crit. & 4,211.735 \\ 
Bayesian Inf. Crit. & 4,244.241 \\ 
\hline 
Var: participant|group (Intercept) & 3.86          \\
Var: group (Intercept)           & 0.00          \\
Var: Residual                              & 11.77         \\
\hline
\hline \\[-1.8ex] 
\textit{Note:}  & \multicolumn{1}{r}{$^{*}$p$<$0.1; $^{**}$p$<$0.05; $^{***}$p$<$0.01} \\ 
\end{tabular}
}
\end{table} 


\subsection{Optimum of Work Supply}
During the competition, participants tend to work more than their optimum, but decreasingly so with each round. Remember, participants should switch as soon as the time they need to solve a given task surpasses the earnings in the switch mode for the corresponding time. In the control group, players with the low wage should change as soon as they take longer than 10 seconds to solve the task, in the treatment, as soon as they need more than six seconds. In each treatment, players with the high wage should switch at twice the time. Graph X shows the time needed for each task per round and marks the 10 and 20 seconds cutoffs.\\ 

\begin{figure}
    \centering
    \includegraphics[width=\textwidth]{graphs/time_per_task.png}
    \caption{Count of Choice of Switching Row}
    \label{fig:time_per_task}
\end{figure}

Valuation avg \\

\subsection{Cognitive Reflection Test}
In the \textit{Cognitive Reflection Test}, participants answered, out of 4, in average 2.22 (s.e. 0.1) questions correctly. Men answered in average 0.3 (s.e. 0.15) more questions correctly with women giving 2.09 (s.e. 0.13) right answers in average.\\


\subsection{Risk Elicitation Task}
In the \textit{Risk Elicitation} task, participants chose in average row 6.6 to switch which according to table \ref{table:HL} indicates a risk averse to very risk averse behaviour. Men generally showed a more risk averse behaviour, switching in average at row 6.93 (s.e. 0.32), while women switched in average already at row 6.36 (s.e. 0.33). Figure \ref{fig:hist_mpl} shows a histogram of the row at which participants switched.\\

\begin{figure}
    \centering
    \includegraphics[width=\textwidth]{graphs/hist_mpl.png}
    \caption{Count of Choice of Switching Row}
    \label{fig:hist_mpl}
\end{figure}

\subsection{Fairness Sentiment}

Participants did not seem to change their views on the fairness of the game. Figure \ref{fig:fairness_boxplot} shows a boxplot of their assessment at the beginning and at the end of the competition rounds. Table \ref{tab:fair_ols} show a linear regression that shows no apparent significant influence in changing a participant's view on fairness.\\

\begin{figure}
    \centering
    \includegraphics[width=\textwidth]{fairness_sentiment_boxplot.png}
    \caption{Fairness Sentiment Voting}
    \label{fig:fairness_boxplot}
\end{figure}



\begin{table}[!htbp] \centering 
  \caption{OLS Change in Fairness Sentiment} 
  \label{tab:fair_ols} 
\begin{tabular}{@{\extracolsep{5pt}}lc} 
\\[-1.8ex]\hline 
\hline \\[-1.8ex] 
\\[-1.8ex] & Change in Fairness Sentiment \\ 
\hline \\[-1.8ex] 
 treatment & $-$0.183 \\ 
  & (0.391) \\ 
  & \\ 
 payoff & 0.015 \\ 
  & (0.052) \\ 
  & \\ 
 crt score & 0.063 \\ 
  & (0.188) \\ 
  & \\ 
 switching row & 0.013 \\ 
  & (0.084) \\ 
  & \\ 
 gender (Male) & $-$0.181 \\ 
  & (0.366) \\ 
  & \\ 
 valuation & 0.048 \\ 
  & (0.063) \\ 
  & \\ 
 Constant & $-$0.971 \\ 
  & (1.298) \\ 
  & \\ 
  \hline
Observations & 96 \\ 
R$^{2}$ & 0.016 \\ 
Adjusted R$^{2}$ & $-$0.051 \\ 
Residual Std. Error & 1.733 (df = 89) \\ 
F Statistic & 0.235 (df = 6; 89) \\ 
\hline \\[-1.8ex] 
\textit{Notes:} & \multicolumn{1}{l}{$^{***}$Significant at the 1 percent level.} \\ 
 & \multicolumn{1}{l}{$^{**}$Significant at the 5 percent level.} \\ 
 & \multicolumn{1}{l}{$^{*}$Significant at the 10 percent level.} \\ 
\end{tabular} 
\end{table} 

\section{Hypotheses}

In this section we have a look at the results regarding, one, the distribution of probabilities and its change across rounds, and, second, the factors affecting the probability of winning for a given individual.


\subsection{Hypotheses 1 and 2}

To test this hypothesis, I firstly need to determine the optimal investment value for each participant. Assuming an equal valuation for all participants greatly simplifies the process by allowing to use formula \ref{eq:opt_last}. Running a t-test on the hypothesis that the mean is the same, supports this assumption.

Since the treatment and control groups have different valuations, I analyze rather the mean difference over optimal investment for both groups.\\

Consistent with expectations, participants bid significantly more than their optimum but decreasingly so over time. Interestingly, participants in the taxation treatment invest both nominally and relatively more than those in the control, with treatment participants bidding in average 5 and a half times their optimal investment in the first round and 2.3 times their optimum in the last round. Participants in the control, on the other hand, invest from 1.93 times their optimum in the first round, to 1.36 times in the last round.\\


\begin{figure}
    \centering
    \includegraphics[width=\textwidth]{graphs/over_invest.png}
    \caption{Ratio of Over Investment per Round}
    \label{fig:over_invest}
\end{figure}

Table \ref{tab:over_invest} present a linear mixed model dissecting the most important effects on over investing. The random effects are \textit{participants} nested in the \textit{group}.\\

\begin{table} \centering 
  \caption{Linear Mixed Model Over Investment Ratio} 
  \label{tab:over_invest}
  \scalebox{0.7}{%
\begin{tabular}{@{\extracolsep{5pt}}lcc} 
\\[-1.8ex]\hline 
\hline \\[-1.8ex] 
\\[-1.8ex] & \multicolumn{2}{c}{Ratio of Investment Over Optimum} \\ 
\\[-1.8ex] & (1) & (2)\\ 
\hline \\[-1.8ex] 
 treatment & 2.314$^{***}$ & 2.075$^{***}$ \\ 
  & (0.698) & (0.678) \\ 
  & & \\ 
 round & $-$0.307$^{***}$ & $-$0.307$^{***}$ \\ 
  & (0.055) & (0.055) \\ 
  & & \\ 
 was winner & 0.256 & 0.281 \\ 
  & (0.271) & (0.271) \\ 
  & & \\ 
 CRT score &  & $-$0.183 \\ 
  &  & (0.289) \\ 
  & & \\ 
 switching row &  & $-$0.375$^{***}$ \\ 
  &  & (0.126) \\ 
  & & \\ 
 gender (Male) &  & 0.284 \\ 
  &  & (0.583) \\ 
  & & \\ 
 Constant & 2.813$^{***}$ & 5.689$^{***}$ \\ 
  & (0.546) & (1.097) \\ 
  & & \\ 
  \hline
  \hline
Observations & 672 & 672 \\ 
Log Likelihood & $-$1,751.726 & $-$1,747.753 \\ 
Akaike Inf. Crit. & 3,517.453 & 3,515.506 \\ 
Bayesian Inf. Crit. & 3,549.024 & 3,560.608 \\
\hline
Var: participant|group (Intercept) & 5.66 &     5.17     \\
Var: group (Intercept)           & 1.62  &    1.48    \\
Var: Residual                              & 8.14   &   8.14    \\
\hline \\[-1.8ex] 
\textit{Notes:} & \multicolumn{2}{l}{$^{***}$Significant at the 1 percent level.} \\ 
 & \multicolumn{2}{l}{$^{**}$Significant at the 5 percent level.} \\ 
 & \multicolumn{2}{l}{$^{*}$Significant at the 10 percent level.} \\ 
\end{tabular}
}
\end{table}

The first and second models have a total explanatory power of around 52\% (conditional r-squared) in which the fixed effects of the first explain 10.03\% and the second 14.5\% of the variance (marginal r-squared). The effect of the treatment is significant and large for both models with participants investing, contrary to hypothesis \ref{hyp:treat-overinvest} almost double as much as the control. Round repetition and risk averse behavior have both a significant although relatively small effect in an expected negative direction. The CRT score and whether a participant was playing with a high wage did not have a significant effect on the over investment ratio.\\

A visual analysis of the residuals, as well as several normality test, strongly reject the assumption of normality.\\

A possibility is that participants apply a heuristic of a fix proportion of available income for investment. Running a similar regression as above on the investment ratio of available income in fact deletes the effect of all fixed effects but time, suggesting that participants tend to invest the same amount of their available income, regardless of their earnings or idiosyncratic features.\\

A Wilcoxon rank test suggest that in fact total welfare is larger in the treatment than in the control. Higher welfare in the treatment seems to be driven by working closer to the optimum while at the same time investing less.

\subsection{Hypothesis 3}

In terms of probabilities, I am interested in how its distribution change. Of course, the mean probability of earning the high wage is always equal to $1/n$, but its distribution can vary from exactly $1/n$ for each participant, to only one person in the group having a probability of 1 and all the rest a probability of 0. Figure \ref{fig:dens_prob} shows the density plots for each round. A completely equal distribution would have its maximum at 0.33 with no values around it, while a completely unequal distribution would see two peaks at each extreme with the one at 0 being twice the size of the the one at 1.\\

\begin{figure}[H]
    \centering
    \includegraphics[width = \textwidth]{graphs/density_ridge_prob.png}
    \caption{Density Plots of Probability across Rounds}
    \label{fig:dens_prob}
\end{figure}

Figure \ref{fig:dens_prob} seems in fact to show an increase of density at the ends of the distribution with round 1 having a rather flat shape and round 8 stronger peaks at both 0 and 1 probability. To test if the distribution of probabilities between the treatments really differ, I use the coefficient of variation as for instance in XXXX, It is a measure widely used in economics, sociology and natural sciences, and as XXX recommends, appropiate to use in cases where there is a relatively exact measure of the values. Figure \ref{fig:var_coeff_boxplot} shows a box plot of the coefficient of variations across rounds for both treatments. In fact, it seems to increase across rounds but not significantly so between treatments.\\

\begin{figure}[H]
    \centering
    \includegraphics[width = \textwidth]{graphs/var_coeff_prob_boxplot.png}
    \caption{Boxplot Coefficient of Variation}
    \label{fig:var_coeff_boxplot}
\end{figure}

Table \ref{tab:var_coeff_ols} shows the results for an OLS regression of two models explaining the difference across groups in the coefficient of variation for the probabilities of winning. A linear mixed model is left out here since each group is part of only one treatment and the variables are all aggregated at the group level. Only round repetition appears to have a significant, although small effect (in our setting, the coefficient of variance can fluctuate between 0 and 1.73). Idiosyncratic group values like the mean value of valuation, CRT score or risk aversion do not play a significant role in the variation of probabilities. Taxation and redistribution also do not play a role. In other words, introducing a redistribution scheme does not avoid the fact that, with time, the distribution of probabilities will go to extremes with some participants having most of the chance to win and the rest having almost none.\\


\begin{table}[!htbp] \centering 
  \caption{OLS Coefficient of Variation} 
  \label{tab:var_coeff_ols} 
\begin{tabular}{@{\extracolsep{5pt}}lcc} 
\\[-1.8ex]\hline 
\hline \\[-1.8ex] 
\\[-1.8ex] & \multicolumn{2}{c}{Probability Coefficient of Variation} \\ 
\\[-1.8ex] & (1) & (2)\\ 
\hline \\[-1.8ex] 
 treatment & $-$0.025 & $-$0.031 \\ 
  & (0.063) & (0.065) \\ 
  & & \\ 
 round & 0.043$^{***}$ & 0.043$^{***}$ \\ 
  & (0.016) & (0.016) \\ 
  & & \\ 
 mean CRT score &  & 0.092 \\ 
  &  & (0.059) \\ 
  & & \\ 
 mean switching row &  & $-$0.020 \\ 
  &  & (0.025) \\ 
  & & \\ 
 mean valuation &  & $-$0.033 \\ 
  &  & (0.023) \\ 
  & & \\ 
 Constant & 0.856$^{***}$ & 1.218$^{***}$ \\ 
  & (0.077) & (0.338) \\ 
  & & \\
  \hline
Observations & 224 & 224 \\ 
R$^{2}$ & 0.034 & 0.053 \\ 
Adjusted R$^{2}$ & 0.025 & 0.031 \\ 
Residual Std. Error & 0.468 (df = 221) & 0.467 (df = 218) \\ 
F Statistic & 3.857$^{**}$ (df = 2; 221) & 2.431$^{**}$ (df = 5; 218) \\ 
\hline \\[-1.8ex] 
\textit{Notes:} & \multicolumn{2}{l}{$^{***}$Significant at the 1 percent level.} \\ 
 & \multicolumn{2}{l}{$^{**}$Significant at the 5 percent level.} \\ 
 & \multicolumn{2}{l}{$^{*}$Significant at the 10 percent level.} \\ 
\end{tabular} 
\end{table}

Of course, this does not tell us much about what determines the probability of a given participant to win a contest. To analyze it, I suggest again a linear mixed model although this time a generalized mixed model from a beta distribution. Table \ref{tab:glm_prob} shows the results.\\

The only variable that shows a significant, although small, effect on the probability of winning is the size of the available income. Note in particular, that neither the treatment, nor having won alone have significant effect on the probability of winning.\\

\begin{table}[!htbp] \centering 
  \caption{GLMM Probability of Winning} 
  \label{tab:glm_prob} 
\begin{tabular}{@{\extracolsep{5pt}}lcc} 
\\[-1.8ex]\hline 
\hline \\[-1.8ex] 
\\[-1.8ex] & \multicolumn{2}{c}{Probability of Winning} \\ 
\\[-1.8ex] & (1) & (2)\\ 
\hline \\[-1.8ex] 
 treatment & $-$0.139 &  $-$0.087241 \\ 
  & (0.22) & (0.218) \\ 
  & & \\ 
 available income & 0.045$^{***}$ & 0.045$^{***}$ \\ 
  & (0.008) & (0.008) \\ 
  & & \\ 
 was winner & $-$0.084 &  $-$0.083 \\ 
  & (0.119) & (0.119) \\ 
  & & \\ 
 gender (male) &  & 0.356 \\ 
  &  & (0.223) \\ 
  & & \\ 
 CRT Score &  & $-$0.051 \\ 
  &  & (0.113) \\ 
  & & \\ 
 switching row & & $-$0.034 \\ 
  &  & (0.049) \\ 
  & & \\ 
 valuation &  & 0.056 \\ 
  &  & (0.038) \\ 
  & & \\ 
 mean investment belief &  & 0.003 \\ 
  &  & (0.008) \\ 
  & & \\ 
 Constant & $-$1.413$^{***}$ & $-$2.454$^{***}$ \\ 
  & (0.209) & (0.647) \\ 
  & & \\ 
  \hline
Observations & 672 & 672 \\ 
Log Likelihood & 3798.4 & 3801.0 \\ 
Akaike Inf. Crit. & $-$7580.9 & $-$7575.9 \\ 
Bayesian Inf. Crit. & $-$7544.8 & $-$7517.3 \\
\hline
Num. groups: participant:group     & 96     &   96   \\
Num. groups: group               & 32      &    32  \\
Num. groups: round                         & 7   & 7        \\
\hline
Var: participant:group (Intercept) & 0.93   & 0.02     \\
Var: group (Intercept)           & 0.00   & 0.00     \\
Var: round (Intercept)           & 0.00    & 0.00    \\
Var: Residual                    & 0.00     &  0.02  \\
\hline
\hline \\[-1.8ex] 
\textit{Notes:} & \multicolumn{2}{l}{$^{***}$Significant at the 1 percent level.} \\ 
 & \multicolumn{2}{l}{$^{**}$Significant at the 5 percent level.} \\ 
 & \multicolumn{2}{l}{$^{*}$Significant at the 10 percent level.} \\ 
\end{tabular} 
\end{table} 