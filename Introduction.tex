\chapter{Introduction}
\thispagestyle{fancy}
\label{ch:intro}


Success depends, in most circumstances in life, not only on one's effort but on that of others. Take, for instance, patent races, litigation, military combat, or sports competitions, where the winner is -- with varying degrees of luck involved -- the one who exerted the most effort or spent the largest amount of money in comparison to the rest \citep{konrad2009}. Competitions of this sort are commonly modelled in social sciences as rent-seeking contests, as first described by \cite{tullock1980}  and expressed in a general form by \cite{sheremeta2010a}. \\

Rent-seeking contests, also called Tullock contests, are contests in which the probability of winning is a function of the relative effort exerted. In their most basic form, two players compete for a prize that is of equal value to both. Participants can place a simultaneous, non-refundable investment. Each of the players is then assigned a probability of winning equal to their respective share of the total investments. This would mean that if both make the same investment, each will have a 50\% chance of winning the prize. Yet, if one invests twice the amount of the other, he or she will have a 66.67\% chance of winning.\\ 

In this thesis I use rent-seeking contests as the base model for an economic experiment in order to better understand how inequality affects competitive behaviour in human capital investments, and how it relates to the process of social mobility. In particular, I focus on whether this type of contest contributes to the perpetuation of inequalities and what the impact is on overall welfare.\\

With Tullock type contests, behavioral economists can model and analyze individual and social behavior in a competitive environment, and how it affects decision making. For instance, while an optimal investment strategy for this game exists, empirical examinations show that people tend to bid too much and therefore incur great efficiency losses with a median overbidding rate of 72\% \citep{sheremeta2013, chowdhury2014, konrad2009, dechenaux2015}.\\

One of the several explanations for overbidding suggested in the literature on this topic is that winning itself has an inherent utility, separate from the utility of the prize. \cite{sheremeta2010} showed this in an experiment in which participants could invest and participate in the game, but winning did not carry a prize of positive monetary value. The prize was merely symbolic. In that study, 40\% of players invested a positive amount of money to win the prize of value 0. Moreover, participants more likely to invest in that setting, were also more likely to over-invest when a prize of positive value was offered.\\

Another common explanation is bounded rationality. Participants can make mistakes, or be confused about the rules or strategy of the game. \cite{sheremeta2011} shows, using the \textit{quantal response equilibrium} (QRE) developed by \cite{mckelvey1995}, that overbidding behaviour is consistent with bounded rationality. Subjects' bids are dependent on their initial endowment amounts, rather than on the stakes of the game. While keeping the award value constant, increasing or decreasing the endowment of participants also increases or reduces the efforts of the players.\\ 

The way in which information about other's investments and distribution levels influence a subject's investments during a rent-seeking contest was studied by \cite{fallucchi2013}. In their research they find that when given full information about their own and other's decisions, over-spending by participants decreases over time to around 13\% above the optimum bid. In comparison, participants who only knew about their own efforts kept their investments high at approximately 63\% over the optimal level.\\

Accounting for several of these explanations simultaneously, in one of his most recent studies, \cite{sheremeta2016} showed that while overbidding is correlated with bounded rationality, systematic biases, the utility of winning and relative payoff maximization, only impulsivity remains significant in a joint multivariate analysis. Impulsive behavior was measured using the Cognitive Reflection Test (CRT) developed by \cite{frederick2005}. It aims to measure the tendency of a person to \textit{stop and think} about a question with a seemingly trivial and intuitive, but ultimately incorrect, answer.\footnote{A detailed description and its use in an experimental setting is given in section \ref{ss:CRT}.}\\

\section{Social Mobility as a Rent-Seeking Contest}
\label{sec:soc_mob}

Understanding the determining aspects of individual behaviour in rent-seeking contests allows us to model more complex dynamics in economics. Social mobility, for example, can be seen in many ways as a rent-seeking contest: in order to obtain a high salaried position, society members exert effort through work supply and investment in human capital. Given the limited nature of those positions, however, only a few are rewarded with the prize \citep{burkhauser2011}.\\

The results listed above might lead us to think that people tend to invest too much in their education or put too much effort into their careers. Yet, there is one crucial way in which social mobility in real life differs from a Tullock contest: the outcome of the contest at a certain stage is likely to depend on the result of the previous stage. In other words, social mobility in practice is likely to be a series of contests in which those who have won previous stages are more likely to win the next. Illustrating this, it can be seen that alumni of respected schools are more likely to access the best universities, and alumni of those universities are more likely to obtain better jobs, thus creating inequalities that may appear insurmountable \citep{sewell1971}.\\

A thorough understanding of how the inequality arising from a previous performance drives an individual's behavior is key to shaping redistribution and equality policies. Being two of the most important fields of research in economics, the literature on social mobility and inequality is, in equal parts, massive and diverse \citep{nolan2011, atkinson2015, lipset2018, fields1999}. In this thesis, I restrict myself to a model of intra-generational social mobility with only one high income position.\\ 

While there is extensive research on the effect of inequality on economic outcomes, such as productivity \citep{persson1994, ku2012}, provision of public goods \citep{fehr1999} and growth \citep{ehrhart2009}, there is limited literature on how inequality affects a single agent's behaviour in a competitive setting. Does, for instance, a worker with lower chances of obtaining a promotion reduce his or her output altogether and are they thus less likely to invest in human capital? Or does he or she invest \textit{too much} given the expected earnings? In other words, how does inequality of opportunity affect work supply and social mobility in a competitive setting?\\

\cite{fallucchi2017}, in a laboratory setting, studied the efforts of participants of a Tullock contest in the presence of different sources of inequality: endowment, ability, and outcomes. In all treatments participants played 30 rounds of a lottery contest. At the beginning of each round, players received an endowment of tokens that they could trade for lottery tickets, the probability of winning the round being equal to their respective share of tickets bought. The prize in tokens -- plus the tokens not used to buy tickets -- were stored in a separate account and paid out at the end of the experiment.\\

The baseline was a contest in which players received 95 tokens of endowment, played for a prize of 80 tokens, and where each lottery ticket costed one token. In contrast, in the treatment scenario with \textit{inequality of endowments} one of the contestants in the group received 120 tokens while the other received only 80. In the group with \textit{inequality of ability} one of the players in each couple got one ticket per token, while the other received three. Finally, in the \textit{inequality of outcomes} treatment, the prize was of 120 tokens for one of the subjects but 40 for the other. In each of the treatments, apart from the inequality parameters, all conditions remained constant.\\

In their study, \citeauthor{fallucchi2017} find evidence that the source of inequality determines the magnitude of competitive effort put in, with participants in groups with unequal abilities making larger investments when compared to the benchmarking case. Subjects in groups with unequal endowments or prizes, on the other hand, invested less than their counterparts in the control group. Most of these effects were driven by the disadvantaged subjects. Participants with \textit{lower} ability displayed higher effort than advantaged players, and, while disadvantaged subjects in the \textit{inequality of outcomes} treatment reduced their efforts, advantaged players did not increase theirs accordingly. In the same manner, an investment gap was observed in the treatment with different endowments, even though theory would predict no difference.\\

Importantly, however, the source of inequality in \citeauthor{fallucchi2017}'s experiment (\citeyear{fallucchi2017}) is exogenous. Participants are awarded more or less endowment or a higher or lower valuation at random. But one can argue that, in the case of social mobility, inequality arises at least in part from success in previous performance. More diligent students get better opportunities in the job market and better employees are awarded higher incomes. Luck still plays a role, of course, and people who put in less effort might still win the prize. But how does it affect overall performance if the source of inequality is at least, in theory, meritocratic? And does success have a tendency to perpetuate itself? In this present thesis, I aim to answer these questions by designing and conducting an experiment in which participants of a repeated rent-seeking contest can earn their endowments by solving a \textit{real effort task}, and subsequently invest their earnings to increase their wages in the subsequent round of the task. The main manifestation of inequality studied here is the difference in winning probabilities between members of a group. Specifically, I pose the following research question:\\ 


\begin{tcolorbox}[colback=UniVieGrau!15!white, colframe=white] 
\textbf{Research Question 1:} How does the distribution of winning probabilities in repeated rent-seeking contests develop over time when the endowments available for investment are determined by previous success and work supply?
\end{tcolorbox}

Furthermore, assuming that inequality in endowments has a negative impact on the work supply and effort provision of participants, one might suggest an intervention designed to distribute those endowments more equally. To analyze the impact of such a scheme, the experiment I conducted includes a treatment which taxes and redistributes the income earned in the \textit{real effort task} among the participants. The following research question thus complements the first:\\

\begin{tcolorbox}[colback=UniVieGrau!15!white, colframe=white] 
\textbf{Research Question 2:} How does a taxation and redistribution scheme affect work supply and effort provision in rent seeking contests like those described in Research Question 1?
\end{tcolorbox}

As I am aware that many factors play a role in how people behave in competitions and how that may impact the results, I designed the experiment and selected model parameters in a way that presents sensible benchmarks and allows future researchers to easily change values, extend the model and devise new interventions.\\ 

\section{Rent-Seeking Contest as Prospect of Upward Mobility}
\label{sec:poum}

To better illustrate the socio-economic implications of the research questions, it helps to see the process of social mobility modeled in the previous section as analogous to a simplified version of the Prospect of Upward Mobility (POUM): the probability of achieving (or staying with) an income above average in the following period, formalized by \cite{benabou2001}. During the experiment presented in this thesis, subjects participate in several rent-seeking contests where the prize is a wage above average in the following round. The probability of a player of winning one of those contest is thus equal to his or her Prospect of Upward Mobility.\\

In their research, \cite{benabou2001} primarily focus on the influence that the POUM has on citizens' voting behaviour. They show, through theoretical considerations, that when individuals decide between taxation schemes, they not only consider their current social standing, but also their probability of obtaining a higher social position in the future. This helps to account for the seemingly contradictory behaviour of low-income citizens voting for a reduction of taxes for higher incomes and thus against redistribution as a whole.\\

According to \citeauthor{benabou2001}, the extent to which citizens approve or disprove of redistribution depends in particular on the time validity of taxation and the shape of the social mobility probability function. Theorems 1 and 2 from their theoretical analysis best summarize their findings:\\
\begin{enumerate}
    \item \textit{Theorem 1:} The more concave the transition function is, the less support there is for redistribution among low income voters. The concavity of the transition function expresses the increasing odds of obtaining a higher income in the future with increasing current income, but at a decreasing rate.  
    \item \textit{Theorem 2:} The more "sticky" taxation policy is, or the more forward-looking citizens are, the less support there is for redistribution.
\end{enumerate}

\cite{checchi2003} tested this idea in the laboratory and found strong support for it, especially the two theorems above, even after controlling idiosyncratic factors like risk and inequality-aversion. In their experiment, subjects were randomly placed into one of three income levels and had to decide on the taxation levels of future income based on the probability of going from one level of income to the other. Furthermore, depending on the treatment, the taxation chosen would be applicable to one, three, or five periods.\\

The results show that for longer periods of valid taxation, the average tax chosen by the participants declined, thus supporting the second theorem of \citeauthor{benabou2001}. In the same manner, and in line with the first theorem, when subjects were confronted with a more mobile society, i.e. the probability of transition from a lower income level to a higher one increased, they were more likely to choose a lower taxation rate.\\

An important implication of the POUM Theorem in the context of this thesis is that once a taxation scheme is in place, the POUM might be altered after redistribution if the probability of improving one's social standing depends on the redistribution result. Imagine, for instance, a scenario in which the ability to invest in the training needed to increase your income depends on your previous income. If the distribution of income changes with each period, voting in favor or against redistribution will change accordingly, mitigating or aggravating inequality. Furthermore, if for some subjects the probability of achieving a given status becomes easier with each passing period while it becomes more difficult for others, a poverty trap can arise through complacency or resignation at the extremes of the income distributions \citep{ceroni2001, ku2012}.

\begin{center}
  $\ast$~$\ast$~$\ast$
\end{center}

To recapitulate: in this thesis I design and conduct an experiment based on rent-seeking contests to study the effect of inequality on work supply and investment in human capital within an imperfectly meritocratic society. Inequality, expressed as the difference in the Prospect of Upward Mobility within a group, is made endogenous to the model by way of work supply in a real effort task, and monetary investments in a Tullock contest.  Furthermore, I study how the implementation of a taxation and redistribution scheme promotes or suppresses the observed behaviour.\\

In a nutshell, the process of social mobility, as modelled and implemented in the experiment, is as follows: subjects work for an endowment they can invest in a rent-seeking contest. The contest grants the winner a higher wage for the following period of the real effort task, thus granting a higher endowment in case of constant work supply. In the intervened group the incomes from solving the \textit{real effort task} were taxed and redistributed within the group, making the endowments for investment more equally distributed.\\

This thesis is presented in six parts. This introductory chapter outlined the study and offered an overview of previous research, upon which this thesis builds. Additionally, it postulated the research questions. Chapter \ref{ch:model} maps out the theoretical foundation of the experiment with a formal model that generalizes \cite{koch2017}, and derives mathematical predictions that build the base of the hypotheses in section \ref{sec:hyp}. Chapter \ref{ch:experiment} presents the design and procedure of the experimental sessions, while chapter \ref{ch:results} deals with the results and their analysis. Finally, chapters \ref{ch:discussion} and \ref{ch:conclusion} conclude with policy implications, suggestions for extended research, and a short summary.