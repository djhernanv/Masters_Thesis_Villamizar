\chapter{Introduction}
\thispagestyle{fancy}
\label{ch:intro}


The present thesis lays out the design and implementation of a laboratory experiment evaluating the interplay of taxation, income inequality and the Prospect of Upward Mobility. :::I find that, consistent with other work, people over invest when participating in a rent-seeking contest. Inequality has a … effect on this overinvestment but forcing it trough taxation does not seem to correct it.::: \\

In several circumstances in life success does not only depend on one's effort but also on that of others. Take for instance promotions in companies or sports competitions, where the winner is the one who exerted the most effort in comparison to the rest. Competitions of this sort are commonly modeled in human sciences as rent-seeking contests, as first described by \cite{tullock1980}. In those contests the probability of winning is a function of the relative effort exerted (How this functions work specifically is explained in more detail in section \ref{ch:model}).\\

Similarly, social mobility in terms of increasing income can be seen as such a rent-seeking contest given that there are only limited high earning positions. Society members exert effort trough work and investment in human capital, but only few are rewarded with the prize of higher income. That means that citizens can actively increase or decrease their chances of improving their social standing by exerting more or less effort.\\

Those chances of winning a higher income and therefore of improving one's social standing can be seen as analogue to the Prospect of Upward Mobility (POUM) of \cite{benabou2001}. In their research they concentrate on the influence that POUM has on citizens' voting behaviour. They showed that when individuals decide between taxation schemes, they do so not only depending on their current social standing, but also on their probability of obtaining a higher social position in the future -- their Prospect of Upward Mobility (POUM). Which helped explain the seemingly contradictory behaviour of low-income citizens voting for a reduction of taxes for higher incomes and thus of redistribution as a whole.\\

Conversely, once a taxation scheme is in place, the POUM might be altered after the redistribution period if the probability of improving one's social standing depends on the redistribution result. Imagine for instance the case were investing in the training needed to rise your income depends on your previous income. Such a situation can create complacency or resignation at the extremes of the income distribution since for some the probability of achieving a given status becomes easier while it becomes more difficult for others.\\

To analyze this process and to test if redistribution has an effect on work supply and human capital investment, and how it alters the dynamic of the POUM, I devised an experiment where participants have the possibility to work and invest their net earnings in order to increase their probability of obtaining a higher income in the next period. The experiment and the model behind it are designed in such a way that it allows for several variations of treatments. In the current thesis I focus on the impact that a taxation and redistribution scheme would have on work supply and the invested amount.\\

The text is presented in five parts. The introduction in chapter \ref{ch:intro} motivates the thesis and offers an overview of previous research. Chapter \ref{ch:model} lays the theoretical foundation of the experiment with a formal model that generalizes \cite{koch2017}, and derives mathematical predictions that build the base of the hypotheses in section \ref{ss:hyp}. Chapter \ref{ch:experiment} presents the design and procedure of the experimental sessions, while chapter \ref{ch:results} deals with the results and their analysis. Finally, chapter \ref{ch:conclusion} concludes with policy implications, and suggestions for extended research.\\

\section{Related Literature}

\cite{ok2000} proposed the idea of the Prospect of Upward Mobility as the determining factor behind the seemingly contradictory behavior of low-income citizens voting against higher taxes for the rich and therefore against an increase in social transfers.\\
Theorems 1 and 2 from their theoretical analysis best summarize their findings:
\begin{enumerate}
    \item \textit{Theorem 1:} The more concave the transition function is, the less support for redistribution among low income voters there is. Where the concavity of the transition function expresses the increasing odds of obtaining a higher income in the future with increasing current income, but at a decreasing rate.  
    \item \textit{Theorem 2:} The more "sticky" taxation policy is, or the more forward looking citizens are, the less support for redistribution there is.
\end{enumerate}

\cite{checchi2003} tested this idea in the laboratory and found strong support for it, in particular in regard to the two theorems above, even after controlling for idiosyncratic factors like risk and inequality-aversion. Treatments were the concavity of the transition function, the knowledge of personal income and the degrees of social mobility and of inequality. In contrast, in my thesis the taxation and redistribution scheme is fixed and lets the POUM, the degree of inequality and the degree of social mobility, fluctuate.\\

To simulate situations in which effort provision relative to others' increases the chance of a higher social position in the future I resort to a rent seeking contest as first postulated by \cite{tullock1980} and expressed in a general form by \cite{sheremeta2010a}. As an example of such a process take for instance access to higher education or a promotion within a company. Most of the time, several people compete for the same position and the final result will depend on the performances of them all.\\
Empirical examinations of rent seeking contests have shown, that people tend to overbid and therefore incur great efficiency losses \citep{sheremeta2016, chowdhury2014, konrad2009, dechenaux2015}. Importantly, \cite{sheremeta2016} shows that impulsivity as measured by a Cognitive Response Test, is the strongest predictor of overbidding.\\