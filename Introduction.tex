\chapter{Introduction}
\thispagestyle{fancy}
\label{ch:intro}

\textcolor{red}{(example: athlete contracts in team sports)}
In several circumstances in life success does not only depend on one's effort but also on that of others. Take for instance promotions in companies, or sports competitions, where the winner is the one who exerted the most effort in comparison to the rest. Competitions of this sort are commonly modeled in human sciences as rent-seeking contests, as first described by \cite{tullock1980}  and expressed in a general form by \cite{sheremeta2010a}. Rent-seeking contests are contests where the probability of winning is a function of the relative effort exerted. As an example for such a process take for instance access to higher education or a promotion within a company. Most of the time, several people compete for the same position and the final result will depend on the performances of them all.\\

Empirical examinations of rent seeking contests have shown that people tend to overbid and therefore incur great efficiency losses \citep{sheremeta2016, chowdhury2014, konrad2009, dechenaux2015}. Importantly, \cite{sheremeta2016} shows that impulsivity as measured by a Cognitive Response Test, is the strongest predictor of overbidding.\\

Analogously, social mobility in terms of achieving a relatively higher income (as in \cite{fields1996} or \cite{fields1999}) can be seen as such a rent-seeking contest where several people compete to obtain a restricted high earning position. Society members exert effort trough work supply and investment in human capital, but only few are rewarded with the prize of higher income. That means that citizens can actively increase or decrease their chances of improving their social standing by exerting more or less effort, but only to the extent that other participants do not change their investment.\\

Furthermore, those chances of winning a higher income and therefore of improving one's social standing can be seen as analogue to the Prospect of Upward Mobility (POUM) of \cite{benabou2001}. In their research they concentrate on the influence that the POUM has on citizens' voting behaviour. They showed that when individuals decide between taxation schemes, they do so not only depending on their current social standing, but also on their probability of obtaining a higher social position in the future -- their Prospect of Upward Mobility (POUM). Which specially helped explain the seemingly contradictory behaviour of low-income citizens voting for a reduction of taxes for higher incomes and thus against redistribution as a whole.\\

Theorems 1 and 2 from their theoretical analysis best summarize their findings:
\begin{enumerate}
    \item \textit{Theorem 1:} The more concave the transition function is, the less support for redistribution among low income voters there is. Where the concavity of the transition function expresses the increasing odds of obtaining a higher income in the future with increasing current income, but at a decreasing rate.  
    \item \textit{Theorem 2:} The more "sticky" taxation policy is, or the more forward looking citizens are, the less support for redistribution there is.
\end{enumerate}

\cite{checchi2003} tested this idea in the laboratory and found strong support for it, in particular in regard to the two theorems above, even after controlling for idiosyncratic factors like risk and inequality-aversion. Treatments were the concavity of the transition function, the knowledge of personal income and the degrees of social mobility and of inequality.\\

Conversely, once a taxation scheme is in place, the POUM might be altered after the redistribution period if the probability of improving one's social standing depends on the redistribution result. Imagine for instance the case were investing in the training needed to rise your income depends on your previous income. Since for some the probability of achieving a given status becomes easier while it becomes more difficult for others, a poverty trap can arise trough complacency or resignation at the extremes of the income distributions  \citep{ceroni2001}.\\

Broadly speaking then: citizens might vote to reduce taxation and redistribution, and trough that, further change the POUM on which they decided to vote in the first place. To analyze this process and to test if redistribution has an effect on work supply and human capital investment, and how it alters the dynamic of the POUM, I devised an experiment where participants have the possibility to work and invest their net earnings in order to increase their probability of obtaining a higher income in the next period. In comparison with \cite{ok2000}, then, my design inverts the endogenous/exogenous factors and lets the POUM fluctuate while holding the taxation level constant.\\

Being aware that many factors will play a role in how people behave in competitions and how that translate into their results, I designed the experiment and selected model parameters in a way that presents sensible benchmarks and allows future researchers to easily change values, extend the model and devise new interventions. And while the design and implementation of the experiment is the main focus of the current thesis, I further have a look at the impact that a taxation and redistribution scheme has on work supply and investments on human capital. In that sense, the present thesis has both an exploratory and descriptive character.\\ 

The text is presented in five parts. The introduction in chapter \ref{ch:intro} motivates the thesis and offers an overview of previous research. Chapter \ref{ch:model} lays the theoretical foundation of the experiment with a formal model that generalizes \cite{koch2017}, and derives mathematical predictions that build the base of the hypotheses in section \ref{ss:hyp}. Chapter \ref{ch:experiment} presents the design and procedure of the experimental sessions, while chapter \ref{ch:results} deals with the results and their analysis. Finally, chapter \ref{ch:conclusion} concludes with policy implications, and suggestions for extended research.\\

\textcolor{red}{Notes: Is a shared contest equivalent to our taxed treatment? What would that mean for our hypothesis?\\Control Fracchellis literature in the case of repeated games}
