\thispagestyle{fancy}
\chapter{Discussion and Further Research}
\label{ch:discussion}

Since, to my knowledge, this is the first study of its type, the presented thesis starts from sensible parameters that can be adapted by future researchers to dive deeper into different aspects of the model. In this chapter, I list some of the questions that arose from the results, as well as suggest some design and parameter changes to answer them.\\

Changing, for instance, the difference between the high and low wages could lead to stronger reactions in work supply when winning, and to the sentiment of fairness of the game. Remember from section \ref{sec:budget_constraint} that given a high enough high valuation, losing players will have no possibility to invest their optimal investment. This could furthermore exacerbate the tendency of probabilities to grow apart from each other.\\

In a related manner, but more complex to model, the prize can be increased with each passing round. Alternatively, the contest success function can be altered to be more or less deterministic by changing the exponent $r$ in the equation:

\begin{equation}
    p(I_i,I_{-i}) =
\begin{cases}
    \frac{1}{n},& \text{if } I_i = I_{-i} = 0\\
    \frac{I_i^r}{I_i^r + \sum_{j\neq i}^n I_{j}^r},              & \text{otherwise}
\end{cases}
\label{eq:csf_exp}    
\end{equation}

A result that also pointed the way to a design expansion was the apparent available income share heuristic. In future experiments, one could control, for instance, if participants are aware of how much the prize is worth by eliciting beliefs about the valuation of the game before the investment screen. This would further reveal how much of the over-investment is due to competition properties like last place aversion and winning utility. Likewise, the slider interface for investing could be changed to a number input which could help mitigate the visual effect of choosing a share of the available income instead of a particular value that maximizes expected earnings.\\

A comparison between the present design, which included full information about the participant's own performance as well as the performance of others, and a treatment design with no information could also help to disentangle the impact of learning, a decreasing utility function for winning, as well as competition properties like inequality or last place aversion.\\

In the same manner, a specific workaround to the problem posed in chapter \ref{ss:compt} could have been to add two rounds without an investment or prize option, but with information about the performance of group members. However, this could nevertheless have been used by a participant to send signals about their own valuation of the prize to other participants. Since a participant with a higher valuation would also invest more, sending signals about his or her own valuation might be rewarding even if no prize is to be immediately won. But considering, as shown in section \ref{sec:budget_constraint}, that participants will always be able to invest their optimal amount, the impact of this in the present thesis would have been moderate. Given a high prize, however, the effect would be significant.\\

Other parameter changes that might have a strong repercussion -- the modelling of which is beyond the scope of this thesis -- are, for instance, the number of people in the group or the number of winners per round. Furthermore, one could model the invested amounts as a public good, in a similar way as investment in education works in the real world, such that the investments are either redistributed to the winner(s) or to the entire community.\\

On the policy end, looking back at the original postulates of the Prospect of Upward Mobility, it is difficult to make a prediction as to which way voting would go in a dynamic setting. Considering that the majority of probabilities tended to go to 0, we could expect votes in support of taxing to go up in the long run. On the other hand, as smaller investments also lead to higher probabilities of winning with each subsequent round, taxation could be voted down whenever players have a positive probability of obtaining the prize. Which of the two directions dominates will be strongly dependant on the determinism of the contest expressed through the exponent $r$ of equation \ref{eq:csf_exp} and the valuation of the prize. A design that internalizes both voting on taxation and the POUM would be an interesting way to see how these two factors play into each other, but it would require a better knowledge of the dynamics of several taxation schemes and $r$ parameters beforehand.\\

Finally, an issue of extreme importance that could not be addressed in this thesis is the efficiency of the wins. One could argue that ideally, the participant who wins the contest should be the one who gets the most out of it i.e. the one who earns the most under the high piece rate. This kind of efficiency could also be interpreted as fairness since it not only maximises total welfare but awards the prize to the most deserving participant. It could be perceived as fair, particularly when investments and income are reinvested in the form of a public good. The current design and analysis assumes an equal valuation among participants due in large part to the selection of the real effort task and the relatively low valuation of the prize. Adjusting the difference in wages between winners and losers and calculating the optimal investments according to equation \ref{eq:InvDiffVal} would make it possible to judge the efficiency of the wins and help find interventions that foster fairness.\\

Many other changes and extensions are possible which will hopefully help to better understand the dynamics of Tullock contests in various real-world scenarios, from public education, to the labor market and the provision of public goods.\\

\chapter{Conclusion}
\label{ch:conclusion}

This thesis presented an experimental design aimed at analysing the dynamics of inequality, taxation, and redistribution, in a repeated Tullock Contest with full information. It builds in particular upon the work of \cite{sheremeta2010a}, \cite{fallucchi2017} and \cite{benabou2001} as detailed in the introductory section.\\

%This thesis is presented in six parts. This introductory chapter outlined the study and offered an overview of previous research, upon which this thesis builds. Additionally, it postulated the research questions. Chapter \ref{ch:model} maps out the theoretical foundation of the experiment with a formal model that generalizes \cite{koch2017}, and derives mathematical predictions that build the base of the hypotheses in section \ref{sec:hyp}. Chapter \ref{ch:experiment} presents the design and procedure of the experimental sessions, while chapter \ref{ch:results} deals with the results and their analysis. Finally, chapters \ref{ch:discussion} and \ref{ch:conclusion} conclude with policy implications, suggestions for extended research, and a short summary.

The contest at the core of this thesis differs from other repeated rent-seeking contests in that the groups are kept equal and the prize is expressed in a higher wage for the next working period. In other words, the prize of the contests allows the winner to earn more and increase his or her available income, or to spend more leisure time while keeping earnings constant. This system made income inequality in earnings and in the Prospect of Upward Mobility endogenous to the group. Moreover, in the treatment group, a redistribution scheme was introduced in order to test the effects of taxation on work supply and effort provision.\\

The results of the experimental sessions presented in Chapter \ref{ch:results} both confirmed and contradicted some of the initial hypotheses regarding the distribution of probabilities and the dynamics of investments, as postulated in section \ref{sec:hyp}. Consistent with previous literature, participants across treatments invested more than their respective optimal amounts. In particular, in the experiment I found that participants tend to invest a fixed amount of their available income, regardless of the potential win. Since participants in the taxation treatment have a larger available income but lower expected income, taxed participants invest a relatively larger amount.\\

Although it is not possible to pin-point the mechanism behind this heuristic (anchorage, endowment effect, bounded rationality, etc.), the result suggests that redistributing assets to be invested in Tullock-like contests would increase, rather than decrease, inefficiency by over-investing.\\  

Over time, investment does decrease. This leads, on the one hand, to smaller investments carrying a greater probability of winning and, on the other, to a more skewed distribution of winning probabilities with more and more participants with either a very low or very high chance of obtaining the prize. This effect was observed in both treatments suggesting that the redistribution of income does not necessarily lead to a more equal distribution of opportunity in Tullock-like contests.\\

The overall results of the thesis suggest, in summary, that repeated rent-seeking contests do not necessarily lead to a perpetuation of the winning status, even when winning increases the endowment of participants, as long as they are always able to invest their optimal bid. Furthermore, not only do Tullock-like contests induce great inefficiencies, but the game organiser, the State for instance, could even exacerbate the problem by increasing available income through a redistribution scheme.\\

The impact that the changing distribution of winning probabilities would have on the voting behavior would likely depend on the deterministic index of the contest function. A contest function that awards higher weight to the invested amount would make it more difficult for those with less available income to win. Voters would therefore, according to the Prospect of Upward Mobility of \cite{benabou2001}, be more inclined to implement a taxation and redistribution scheme.\\

From this perspective, programs like student loan schemes can be viewed critically as they induce people to invest more than their expected return. In contrast, programs that tax over-bidding could help reduce inefficiencies by redistributing ``excess'' investments. It would be advisable, however, to redistribute through financing of non-investable assets, like infrastructure, to avoid the problem of increased available income.\\

The study of repeated Tullock contests is essential in understanding the dynamics of multiple group selection processes. Education and work promotions are two of the most prominent examples. With this thesis, I hope to have provided a glimpse into the ramifications of possible design changes and how these can be used to create more fair and more efficient competition.