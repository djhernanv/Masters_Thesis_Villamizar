\chapter*{Abstract}

The present thesis lays out the design and implementation of a laboratory experiment evaluating the interplay of taxation and the Prospect of Upward Mobility. The design takes the form of a repeated Tullock contest where participants decide how much to work and can invest part of their earnings to increase their chances of obtaining a higher piece rate in the following round.\\

I find that, consistent with previous research, people over-invest when participating in a rent-seeking contest. However, in my design, investments over the optimum decrease over time in all treatments. Against my expectations, this effect appears to be more pronounced in groups where a taxation and redistribution scheme has been implemented, in part because those groups invested nominally and relatively more than the control groups. The main driver of this effect appears to be an income effect created trough the more equally redistributed income and as a result of a fixed share of available income heuristic.\\

Probabilities of Upward Mobility grew apart with each passing round, with more participants having probabilities at the end of the distribution. The probability to win itself was determined exclusively by the available income and not, as expected, by the wage or winning status. Further cementing the idea that participants chose their investment as a share of their available income and not of their valuation of the contest.\\


\textbf{Keywords:} Social Mobility, Laboratory Experiment, Behavioral Economics\\

\textbf{JEL Codes:} 
C92 – Laboratory, Group Behavior; C72 - Noncooperative Games;
J24 - (Labor and Demographic Economics - Human Capital, Skills, Occupational Choice, Labor Productivity)\\