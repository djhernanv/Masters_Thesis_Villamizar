\chapter*{Abstract}

\begin{small}
To better comprehend the impact that inequality and redistribution have on social mobility it is important to understand how they affect individual behaviour in competitive settings conducive to the increase of income. For instance, investments in education or a job promotion.\\

In the present thesis I use a rent-seeking contest to model those situations while accounting for two important characteristics of social mobility. Firstly, competitions that together constitute social mobility are likely to be chained in a way that benefits the winner of previous contests. Secondly, inequality within each contest is at least in part due to a difference in performance and is not entirely random.\\

Using that model, I designed and conducted a laboratory experiment in which participants decide how much to work and how much of their earnings they want to invest in a rent-seeking contest in order to increase their chances of obtaining an income above average in the next period. Furthermore, I introduced a taxation and redistribution scheme for one of the studied groups.\\

I found that, consistent with previous research, people over-invest when participating in a rent-seeking contest, although decreasingly so with each passing round. Subjects in the taxed groups invested relatively and nominally more, but were closer to their optimum level of work supply during the real effort task. The main driver for this effect appears to be a combination of the income effect created through the more equally distributed income and an investment heuristic of a fixed share of available income.\\

In both treatments (the control group and the taxation and redistribution scenario), the number of participants with either very high or very low probabilities of winning increased with each round. Nevertheless, I found little evidence that this event is due to the advantage of previous winners. Instead, it was more likely due to a polarisation of the investments made with more participants making very small investments while others made even larger ones.\\

The obtained results suggest that rent-seeking contests do not necessarily lead to a perpetuation of the winning status, as long as participants are always able to invest their optimal bid. Furthermore, they suggest that policies aimed at directly increasing available income from investment in human capital, such as for student loans, might increase inefficiency and drive up the costs of education above its actual economic value.\\

\end{small}

\begin{footnotesize}

\textbf{Keywords:} Social Mobility, Laboratory Experiment, Behavioral Economics, Tullock Contest\\

\textbf{JEL Codes:} 
C92 – Laboratory, Group Behavior; C72 - Noncooperative Games;
J24 - Labor and Demographic Economics - Human Capital, Skills, Occupational Choice, Labor Productivity.

\end{footnotesize}

\chapter*{Abstract}

\begin{small}
Um die Auswirkungen von Ungleichheit und Umverteilung auf die soziale Mobilität besser zu erfassen, ist es wichtig zu verstehen, wie sich diese auf das individuelle Verhalten in einem Wettbewerbsumfeld auswirken, das der Erhöhung des Einkommens förderlich ist. Beispiele hierfür sind Investitionen in Bildung oder einen beruflichen Aufstieg.\\

In der vorliegenden Masterarbeit verwende ich einen rent-seeking Wettbewerb, um diese Situationen zu modellieren. Berücksichtigt werden dabei zwei wichtige Merkmale der sozialen Mobilität: Zum einen hängen Wettbewerbe, aus denen soziale Mobilität besteht, wahrscheinlich so miteinander zusammen, dass GewinnerInnen früherer Wettbewerbe im nächsten Wettbewerb begünstigt werden. Zum anderen ist die Ungleichheit innerhalb jedes Wettbewerbs zumindest teilweise auf einen Leistungsunterschied zurückzuführen, und nicht völlig zufällig.\\

Anhand dieses Modells entwarf und führte ich ein Laborexperiment durch, bei dem die TeilnehmerInnen entscheiden, wie viel sie arbeiten und welchen Teil ihres Verdienstes sie in einen rent-seeking Wettbewerb investieren wollen, um ihre Chancen auf ein überdurchschnittliches Einkommen in der nächsten Periode zu erhöhen. Darüber hinaus führte ich ein Besteuerungs- und Umverteilungsschema für eine der untersuchten Gruppen ein.\\

Ich konnte feststellen, dass MitspielerInnen in Übereinstimmung mit früheren Untersuchungen zu viel investieren, wenn sie an einem rent-seeking Wettbewerb teilnehmen, wenngleich dies mit jeder Runde abnimmt. Die Subjekte in den besteuerten Gruppen investierten relativ und nominell mehr, waren aber näher an ihrem optimalen Arbeitsangebots-Niveau während der \textit{real effort task}. Die Hauptursache für diesen Effekt scheint eine Kombination aus dem Einkommenseffekt, der durch das gleichmäßiger verteilte Einkommen entsteht, und einer Investitionsheuristik eines festen Anteils des verfügbaren Einkommens zu sein.\\

In beiden Treatments stieg mit jeder weiteren Runde die Anzahl der Teilnehmenden mit sehr kleiner oder sehr großer Gewinnwahrscheinlichkeit. Dennoch fand ich wenig Evidenz dafür, dass dieses Ereignis auf die Begünstigung früherer GewinnerInnen zurückzuführen ist. Vielmehr war es auf eine Polarisierung der getätigten Investitionen zurückzuführen, indem mehr TeilnehmerInnen sehr kleine Investitionen tätigten, während andere sogar noch größere machten.\\

Die erzielten Ergebnisse deuten darauf hin, dass rent-seeking Wettbewerbe nicht notwendigerweise zu einer langfristigen Beibehaltung des  Siegerstatus führen, solange die TeilnehmerInnen stets in der Lage sind, ihr optimales Gebot zu investieren. Darüber hinaus deuten sie darauf hin, dass Maßnahmen, die darauf abzielen das verfügbare Einkommen für Investitionen in Humankapital, zum Beispiel Studiendarlehen, direkt zu erhöhen, die Ineffizienz steigern und die Kosten der Bildung über ihren tatsächlichen wirtschaftlichen Wert hinaus in die Höhe treiben könnten.
\end{small}