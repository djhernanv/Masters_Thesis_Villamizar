\chapter{The Contest Success Function}
\label{ch:model}
\thispagestyle{fancy}

The model my experiment builds upon is inspired by \cite{koch2017} and aims to simulate a process by which the effort of a subject determines his or her likelihood to obtain a higher wage in the subsequent period. \cite{koch2017} proposes a Tullock contest in which the relative work supply determines the probability of winning a higher income in the following period and it is described by the Contest Success Function (CSF):

\begin{equation}
    p(x_i,x_{-i}) =
\begin{cases}
    \frac{1}{n},& \text{if } x_i = x_{-i} = 0\\
    \frac{x_i}{x_i + \sum_{j\neq i}^n x_{j}},              & \text{otherwise}
\end{cases}
\label{eq:csf}    
\end{equation}

\hfill \break

Where $x_i$ is the work supplied by participant $i$ and $n$ the total number of participants. So, if a participant provides, for instance, one fourth of the total work supply of the group, she will have a 25 percent chance of obtaining the high income in the following round.\\

Koch's model, however, has several drawbacks. Most notably, that the probability of winning depends only on work supply and therefore a redistribution scheme would have no impact on the likelihood of obtaining a higher wage. 
%Furthermore, it allows for inequality between players only through the disincentive of taxation (some participants are taxed relatively more because they earn less) but not by differences outcome. 
Also, since the level of optimal work supply depends on the utility function and it, in turn, on the expected wage for the following round, the best response function, and therefore the optimization problem, increases in complexity with each round and is not  analytically solvable \citep{koch2017}.\\ 

To address those issues, I consider a model where the contest success function is dependant on an individual's relative investment in human capital –– similar to investment in private education.\\ 

To be precise, at the end of period $t$, an individual $i$ has the opportunity to invest an amount $I$ of her available income to increase her chances of winning a high income in the next period. Her available income is the income obtained from work after taxation and redistribution. Analogously to the case above, the Contest Success Function (CSF) is therefore defined by:

\begin{equation}
    p(I_i,I_{-i}) =
\begin{cases}
    \frac{1}{n},& \text{if } I_i = I_{-i} = 0\\
    \frac{I_i}{I_i + \sum_{j\neq i}^n I_{j}},              & \text{otherwise}
\end{cases}
\label{eq:csf}    
\end{equation}

\hfill \break

To see how a rational participant would go about deciding her optimal investment, let us consider the most simple case of a game with two periods where the share of total investment at the end of the first period determines the probability of obtaining the prize of a high wage in the second according to the CSF.\\

In the second period, the last in this case, exactly one participant will have a high wage $w_h$ while all others will work with a low wage $w_l$. That means that a participant $i$'s prize, or valuation $\mathbb{V}$ of the game, is equal to the difference in earnings between what she would earn with the high versus with the low wage. At the end of the first period, participant $i$ has to then choose how much to invest in order to maximize the chances of obtaining the prize $\mathbb{V}$.\\

Thus, in period 1, a participant wants to maximize his or her expected profit $\mathbb{E}\Pi_i$ according to the optimization problem:

\begin{equation}
    \underset{I_i}{\text{max}}\quad\mathbb{E}\Pi_i(I_i,I_j) = \frac{I_i}{I_i + \sum_{j\neq i}^n I_{j}}\mathbb{V} - I_i
\label{eq:exp_util}
\end{equation}

Assuming risk-neutrality and that the valuation $\mathbb{V}$ of the higher wage is equal for all participants and therefore that $I_i^{*}=I_{-i}^{*}=I^{*}$ it can be easily shown that the optimal investment amount is:\footnote{A step-by-step derivation can be found in Appendix \ref{ax:derivations}.}

\begin{equation}
    I^{*} = \frac{n-1}{n^2}\mathbb{V}
\label{eq:opt_last}
\end{equation}

\hfill \break 

The optimal amount of investment in the first round will hence depend inversely on the number of participants and will be directly proportional to the value of the prize.\\

\section{Ensuring Opportunity: The Budget Constraint}
\label{sec:budget_constraint}

Whilst this may be straightforward, it is interesting to question what happens if there are several consecutive rounds. In the present design, the probability of obtaining the high income in the next period is independent from the previous period only if the optimal investment amount is higher than the available income. In other words, if a participant who has not won the high wage cannot afford the optimal investment, he would have an increased interest to earn the high wage in previous periods. The budget constraint $\frac{n-1}{n^2}\mathbb{V} \leq \pi(w_l)$ is thus violated whenever $\frac{\pi(w_h)-\pi(w_l)}{\pi(w_l)} > \frac{n^2}{n-1}$.\\

Pilot sessions have shown that in a game with three participants, a high wage of approximately five times the low wage will make it impossible for a low wage participant to invest the optimal amount. I take this into account for the current experimental design and make the high wage only twice as large.

\section{Theoretical Predictions}

Regardless of the repetitions of the game, the expected profit maximizing investment for a given player is constant, as shown in equation \ref{eq:opt_last} since, first, the valuation of the game is constant across periods, and, second, the invested amount only affects the winning probabilities in the subsequent period. In other words, the amount of money invested in the contest should not be dependent on the wage of the player, nor on the history of wins and losses.\\

Furthermore, since work supply does not enter the CSF and participants can always afford to invest their optimum, as explained in section \ref{sec:budget_constraint}, there is no incentive to work more than the profit maximizing amount. 